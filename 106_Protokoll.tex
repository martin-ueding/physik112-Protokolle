% Copyright (c) 2012 Martin Ueding <dev@martin-ueding.de>
%
% vim: spelllang=de
\documentclass[11pt]{article}
\usepackage[T1]{fontenc}
\usepackage[a4paper, left=3cm, right=2cm, top=2cm, bottom=2cm]{geometry}
\usepackage[activate]{pdfcprot}
\usepackage[ngerman]{babel}
\usepackage[parfill]{parskip}
\usepackage[utf8]{inputenc}
\usepackage{amsmath}
\usepackage{amssymb}
\usepackage{color}
\usepackage{epstopdf}
\usepackage{graphicx}
\usepackage{hyperref}
\usepackage{setspace}
\usepackage{units}

\definecolor{darkblue}{rgb}{0,0,.5}
\hypersetup{pdftex=true, colorlinks=true, breaklinks=false, linkcolor=black, menucolor=black, pagecolor=black, urlcolor=darkblue}

\setlength{\columnsep}{2cm}

\newcommand{\arcsinh}{\mathrm{arcsinh}}
\newcommand{\asinh}{\mathrm{arcsinh}}
\newcommand{\ergebnis}{\textcolor{red}{\mathrm{Ergebnis}}}
\newcommand{\fehlt}{\textcolor{red}{Hier fehlen noch Inhalte.}}
\newcommand{\half}{\frac{1}{2}}
\renewcommand{\d}{\, \mathrm d}

\title{physik112 Versuch 106 \\ freie und erzwungene Schwingungen mit Dämpfung}
\author{Martin Ueding}

\begin{document}

\maketitle

\section{Einleitung}

Anhand des Pohlschen Drehpendels untersuchen wir das physikalisch bedeutende
Phänomen der freien und erzwungenen Schwingungen mit Dämpfung. Dabei wollen wir
besonders das Phänomen Resonanz beobachten.

\section{Theorie}

Das Drehpendel besteht aus einem Schwingkörper mit einem Trägheitsmoment $I$
und einer Spiralfeder mit einem Richtmoment $D$. Für den ungedämpften Fall
lautet die Bewegungsgleichung also:
\[ I \ddot{\phi} + D \phi = 0 \]

Dies können wir auch mit der Eigenfrequenz schreiben als:
\[ \ddot{\phi} + \omega_0^2 \phi = 0 \]

Nun kommt noch die Dämpfung hinzu, die hier durch eine Wirbelstrombremse
realisiert ist:
\[ \ddot{\phi} + 2 \beta \dot{\phi} + \omega_0^2 \phi = 0 \]

Diese Bewegungsgleichung wird für leichte Dämpfung gelöst durch:
\[ \phi(t) = \exp(-\beta t) \exp(i \omega t) \]

Dabei ist $\omega$, die neue Winkelgeschwindigkeit, definiert als:
\[ \omega = \sqrt{\omega_0^2 - \beta^2} \] 

In diesem Versuch kommt auch ein externer Treiber zum Einsatz. Dadurch ist das
Kräftegleichgewicht aufgehoben und wir erhalten:
\[ \ddot{\phi} + 2 \beta \dot{\phi} + \omega_0^2 \phi = \mu \cos(\omega t) \]

Dabei ist $\omega$ die erzwungene Winkelgeschwindigkeit durch den Treiber. Nach
der Einschwingphase wird das System diese äußere Frequenz annehmen.

Ist die erzwungene Frequenz $\omega$ nahe an der Eigenfrequenz $\omega_0$, so
kommt es zu einer Resonanz. Das System schaukelt sich so weit auf, wie die
Dämpfung die Energie schluckt, die vom Treiber in das System gegeben wird. Die
Güte $Q = \omega_0 / 2 \beta$ gibt an, um wie viel höher die Amplitude des
Systems im Vergleich zur Treiberamplitude ist.

\section{Aufgaben}

\subsection{Aufgabe 106.A}

Die Maßeinheit des Amplitudenquadrats $\phi^2$ ist die Einheit, in der man den
Winkel misst, zum Quadrat. Wählt man die willkürlichen Einheiten auf der Skala
so hat man diese Einheit in Quadrat. Misst man den Winkel, so hat man
$\unit{rad}^2$.

\section{Aufbau und Durchführung}

\subsection{Aufbau}

Wir haben ein Drehpendel mit einer Schneckenfeder, die mit einem rückteibenden
Drehmoment auf den Drehkörper wirkt. Der Drehkörper kann durch eine
Wirbelstrombremse gebremst werden, dabei kann der Bremsstrom eingestellt
werden. Ein Motor ist noch mit der Feder verbunden und kann so ein zusätzliches
Drehmoment auf das Pendel legen. Damit lässt sich eine bestimmte Frequenz
vorgeben und die Resonanz beobachten.

Zeichnung: \fehlt

%%%%%%%%%%%%%%%%%%%%%%%%%%%%%%%%%%%%%%%%%%%%%%%%%%%%%%%%%%%%%%%%%%%%%%%%%%%%%%%
%                                Durchführung                                %
%%%%%%%%%%%%%%%%%%%%%%%%%%%%%%%%%%%%%%%%%%%%%%%%%%%%%%%%%%%%%%%%%%%%%%%%%%%%%%%

\subsection{Durchführung}

\subsubsection{Eigenfrequenz (Aufgabe 106.a)}
\label{a_durchführung}

Wir sollen die Eigenfrequenz $\omega_0$  für den ungedämpften Fall bestimmen.
Dazu schalten wir die Wirbelstrombremse aus. Wir ziehen das Pendel aus der
Ruhelage auf eine Amplitude und lassen es dort los. Dabei achten wir darauf,
dem Pendel keinen „Schubs“ zu geben. Wir zählen die Perioden und messen die
Gesamtzeit mit der Stoppuhr.

\begin{tabular}{cccr}
	Messung & Perioden & Gesamtdauer & Kommentar \\
	\hline
	1 & $\ergebnis$ & $\ergebnis$ \\
	2 & $\ergebnis$ & $\ergebnis$ \\
	3 & $\ergebnis$ & $\ergebnis$ \\
  \vdots & \vdots & \vdots \\
\end{tabular}

Diese Daten werte ich in §\ref{a_auswertung} aus.

\subsubsection{Dämpfung (Aufgabe 106.b)}
\label{b_durchführung}

Wir sollen für drei verschiedene Dämpfungsstärken die abklingenden Amplituden
messen. Wir stellen die Dämpfung so ein, dass der entsprechende Strom fließt.
Dann lenken wir das Pendel aus und notieren nach jeder Schwingung, wie groß die
Amplitude noch ist.

\paragraph{Spulenstrom $\unit[0.1] A$}

Wir messen mit einem Spulenstrom von $\unit[0.1] A$.

\begin{tabular}{ccr}
	Schwingung & Amplitude $\phi_n$ & Kommentar \\
	\hline
	0 & $\ergebnis$ \\
	1 & $\ergebnis$ \\
	2 & $\ergebnis$ \\
	3 & $\ergebnis$ \\
	4 & $\ergebnis$ \\
	5 & $\ergebnis$ \\
	6 & $\ergebnis$ \\
	7 & $\ergebnis$ \\
	8 & $\ergebnis$ \\
	9 & $\ergebnis$ \\
	10 & $\ergebnis$ \\
	11 & $\ergebnis$ \\
	12 & $\ergebnis$ \\
	13 & $\ergebnis$ \\
	14 & $\ergebnis$ \\
	15 & $\ergebnis$ \\
\end{tabular}

\paragraph{Spulenstrom $\unit[0.3] A$}

Wir messen mit einem Spulenstrom von $\unit[0.3] A$.

\begin{tabular}{ccr}
	Schwingung & Amplitude $\phi_n$ & Kommentar \\
	\hline
	0 & $\ergebnis$ \\
	1 & $\ergebnis$ \\
	2 & $\ergebnis$ \\
	3 & $\ergebnis$ \\
	4 & $\ergebnis$ \\
	5 & $\ergebnis$ \\
	6 & $\ergebnis$ \\
	7 & $\ergebnis$ \\
	8 & $\ergebnis$ \\
	9 & $\ergebnis$ \\
	10 & $\ergebnis$ \\
	11 & $\ergebnis$ \\
	12 & $\ergebnis$ \\
	13 & $\ergebnis$ \\
	14 & $\ergebnis$ \\
	15 & $\ergebnis$ \\
\end{tabular}

\paragraph{Spulenstrom $\unit[0.5] A$}

Wir messen mit einem Spulenstrom von $\unit[0.5] A$.

\begin{tabular}{ccr}
	Schwingung & Amplitude $\phi_n$ & Kommentar \\
	\hline
	0 & $\ergebnis$ \\
	1 & $\ergebnis$ \\
	2 & $\ergebnis$ \\
	3 & $\ergebnis$ \\
	4 & $\ergebnis$ \\
	5 & $\ergebnis$ \\
	6 & $\ergebnis$ \\
	7 & $\ergebnis$ \\
	8 & $\ergebnis$ \\
	9 & $\ergebnis$ \\
	10 & $\ergebnis$ \\
	11 & $\ergebnis$ \\
	12 & $\ergebnis$ \\
	13 & $\ergebnis$ \\
	14 & $\ergebnis$ \\
	15 & $\ergebnis$ \\
\end{tabular}

Diese Daten werte ich in §\ref{c_auswertung} aus.

\subsubsection{Motorfrequenz (Aufgabe 106.d)}
\label{d_durchführung}

Wir ermitteln den Zusammenhang zwischen der Frequenz des Exzenters und der
Motorspannung.

Dazu wählen wir vier geeignete Frequenzen zwischen $\unit[0.1]{Hz}$ und
$\unit[1]{Hz}$ und stoppen die Zeit für mindestens 10 Umdrehungen. Die Spannung
lesen wir vom Voltmeter ab, das in der Spannungsversorgung verbaut ist.

\begin{tabular}{cccr}
	Spannung & Umdrehungen & Zeit & Kommentar \\
	\hline
	$\ergebnis$ & $\ergebnis$ & $\ergebnis$ & \\
	$\ergebnis$ & $\ergebnis$ & $\ergebnis$ & \\
	$\ergebnis$ & $\ergebnis$ & $\ergebnis$ & \\
	$\ergebnis$ & $\ergebnis$ & $\ergebnis$ & \\
\end{tabular}

Diese Daten werte ich in §\ref{d_auswertung} aus.

\subsubsection{Resonanz (Aufgabe 106.e)}
\label{e_durchführung}

Wir messen für zwei verschiedene Dämpfungsströme, einmal $\unit[0.3]A$ und
einmal $\unit[0.5]A$. Dabei messen wir zuerst mit der starken Dämpfung, da dort
die Resonanzkurve breiter ist und wir nicht Gefahr laufen, sie zu verpassen.
Dann können wir bei der zweiten Messung mit weniger Dämpfung die Resonanzkurve
besser auflösen, da wir dann wissen, wo das Maximum ist.

Für einen einzelnen Messpunkt stellen wir eine Spannung am Motor ein. Dann
warten wir, bis sich das System ausreichend eingeschwungen hat, also der
stationäre Zustand erreicht ist. Dann bestimmen wir die Maximalamplitude.

\paragraph{starke Dämpfung}

Wir messen mit dem Dämpfungsstrom auf $\unit[0.5]A$.

\begin{tabular}{ccccr}
	Motorspannung [$\unit V$] & Pendelamplitude & Kommentar \\
	\hline
	$\ergebnis$ & $\ergebnis$ & \\
	$\vdots$ & $\vdots$ & \\
\end{tabular}

\paragraph{schwache Dämpfung}

Wir messen mit dem Dämpfungsstrom auf $\unit[0.3]A$.

\begin{tabular}{ccccr}
	Motorspannung [$\unit V$] & Pendelamplitude & Kommentar \\
	\hline
	$\ergebnis$ & $\ergebnis$ & \\
	$\vdots$ & $\vdots$ & \\
\end{tabular}

Diese Daten werte ich in §\ref{e_auswertung} aus.

%%%%%%%%%%%%%%%%%%%%%%%%%%%%%%%%%%%%%%%%%%%%%%%%%%%%%%%%%%%%%%%%%%%%%%%%%%%%%%%
%                                 Auswertung                                  %
%%%%%%%%%%%%%%%%%%%%%%%%%%%%%%%%%%%%%%%%%%%%%%%%%%%%%%%%%%%%%%%%%%%%%%%%%%%%%%%

\section{Auswertung}

\subsection{Eigenfrequenz (Aufgabe 106.a)}
\label{a_auswertung}

In §\ref{a_durchführung} haben wir die Eigenfrequenz des Pendels einige Male
gemessen. Daraus errechne ich die Eigenfrequenz $\omega_0$ aus den einzelnen
Messwerten $x$:
\[ \omega_0 = \overline x \]

Die Standardabweichung gibt den statistischen Fehler für diese Messung:
\[ \Delta \omega_0 = \frac{\sigma(x)}{\sqrt{N}} \]

\subsection{Dämpfung (Aufgabe 106.c)}
\label{c_auswertung}

Die in §\ref{b_durchführung} gemessenen Amplituden sollen nun halblogarithmisch
dargestellt werden. Dazu generiere ich für jede Messreihe $(x, y, \Delta y)$
Tupel:
\[ (x, y, \Delta y) := \left(n, \ln(\phi_n), \frac{\Delta \phi}{\phi_n}\right) \]

Diese lasse ich mit Gnuplot plotten.

\includegraphics[width=\textwidth]{plot_c.pdf}

Die Geradensteigung errechne ich aus diesen Wertepaaren für jede Gerade jeweils
mit:
\[ \mathrm{Steigung} = -(\beta T) = \frac{K(x, y)}{V(x)} \]

Dabei bezeichnet $K$ die Kovarianz und $V$ die Varianz.

Den Fehler für jede Steigung bestimme ich über:
\[ \Delta (-\beta T) = \frac{\sigma(\phi)}{N} \]

Die drei Steigungen sind \fehlt

Anschließend bestimme ich aus jeder Steigung das Dämpungsverhältnis $K$:
\[ K = \exp(\beta T) \]

Sowie den Fehler:
\[ \Delta K = K \beta T \Delta (-\beta T) \]

Die drei Dämpfungsverhältnisse sind \fehlt

Außerdem bestimme ich die Güte $Q$:
\[ Q = \frac{\pi}{\beta T} \]

Und deren Fehler:
\[ \Delta Q = \frac{\pi \Delta (-\beta T)}{(-\beta T)^2} \]

Die drei Gütefaktoren sind \fehlt

\subsection{Motorfrequenz (Aufgabe 106.d)}
\label{d_auswertung}

Hier werte ich die Daten aus §\ref{d_durchführung} aus. Auf der $x$-Achse ist
die Motorspannung aufgetragen, auf der $y$-Achse die Rotationsfrequenz.

Dabei errechnen sich die Datenpunkte wie folgt:
\[ (x, y, \Delta y) := \left( U, \frac N T, \frac{N}{T^2} \Delta T \right) \]

\includegraphics[width=\textwidth]{plot_d.pdf}

Diese Messdaten fitte ich mit einem linearen Modell mit den Parametern $\eta$
und $\zeta$:
\[ \nu = \eta V + \zeta \]

Die Geradensteigung kann ich mit der Formel $\eta = K(x, y)/V(x)$ errechnen.

Für die Steigung erhalte ich:
\[ \eta = \ergebnis \]

Den Fehler bestimme ich mit:
\[ \Delta \eta = \frac{\sigma (y)}{N} = \ergebnis \]

Und für den $y$-Achsen-Abschnitt:
\[ \zeta = \ergebnis \]

Den Fehler bestimme ich mit:
\[ \Delta \zeta = \overline{x^2} \Delta \eta = \ergebnis \]

\subsection{Resonanz (Aufgabe 106.e)}
\label{e_auswertung}

Die Daten, die in §\ref{e_durchführung} gemessen worden sind, werte ich in
diesem Abschnitt aus. Auch hier generiere ich, einmal für die starke, einmal
für die schwache Dämpfung, Tupel zum Plotten:
\[ (x, y, \Delta y) := \left( \eta U + \zeta, \phi_\mathrm{max}, \Delta \phi \right) \]

\includegraphics[width=\textwidth]{plot_e.pdf}

\subsection{Resonanz (Aufgabe 106.e)}

Ich bestimme die die Güte $Q$ aus dem Abstand $\Delta \nu$ der beiden
Frequenzen, für welche die Amplitude auf das $2^{-0.5}$-fache der
Maximalamplitude herabgesunken ist und der Eigenfrequenz $\nu_0$, bei der die
Maximalamplitude auftritt:
\[ Q = \frac{\nu_0}{\Delta \nu} \]

Dazu lese ich $\nu_0 = \ergebnis$ und $\Delta \nu = \ergebnis$ aus dem Diagram
ab. Als Ergebnis erhalte ich $Q = \ergebnis$.

Darüber hinaus bestimme ich die Güte $Q$ aus dem Verhältnis der Amplituden:
\[ Q = \frac{\phi(\nu = \nu_0)}{\phi(\nu = 0)} = \ergebnis \]

\section{Resultat}

Vergleich der $Q$ \fehlt

\section{Diskussion}

\end{document}
